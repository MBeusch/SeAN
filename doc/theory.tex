\documentclass{article}

\usepackage{cite}
\usepackage{hyperref}

\begin{document}
\title{SeAN, Self-Absorption Numerical \\ Theoretical background}
\author{Udo Gayer}
\date{}
\maketitle

\newpage

\tableofcontents

\newpage

\section{Introduction}
This document describes the theoretical background of nuclear self-absorption experiments relevant for the usage of SeAN. 
A comprehensive basic introduction can be found in the PhD thesis of C. Romig (Ref. \cite{Rom15}).
The notation in all SeAN - related documents is also based on this reference.

One of the first introductions into the formalism was given by F.R. Metzger in 1959 \cite{Met59}.

\newpage

\section{Theory}
\subsection{Doppler shift}

\begin{equation}
\label{breit_wigner}
\sigma_{0 \to i} (E, E_i^{nucl}) = \frac{\pi}{2} \cdot \left( \frac{\hbar c}{E_i^{nucl}} \right)^2 \cdot \frac{2 J_i + 1}{2 J_0 + 1} \cdot \frac{\Gamma_0 \Gamma}{\left( E - E_i^{nucl} \right)^2 + \frac{\Gamma^2}{4}}
\end{equation}

\begin{equation}
\label{doppler_shift}
E_\gamma^{lab}(v_\parallel) = \frac{\sqrt{1 - \left( \frac{v_\parallel}{c} \right)^2}}{1 + \frac{v_\parallel}{c}} \cdot E_\gamma^{nucl}
\end{equation}

\begin{equation}
\label{doppler_shift_inverse}
\frac{v_\parallel}{c} \left( E_\gamma^{lab}, E_\gamma^{nucl} \right) = \frac{1 - \left( \frac{E_\gamma^{lab}}{E_\gamma^{nucl}} \right)^2}{1 + \left( \frac{E_\gamma^{lab}}{E_\gamma^{nucl}} \right)^2}
\end{equation}

\begin{equation}
\frac{v_\parallel}{c} \left( 0, E_\gamma^{nucl} \right) = 1
\end{equation}

\begin{equation}
	\frac{v_\parallel}{c} \left( E_\gamma^{nucl}, E_\gamma^{nucl} \right) = 0
\end{equation}

\begin{equation}
\mathrm{lim}_{E_\gamma^{lab} \to \infty} \frac{v_\parallel}{c} \left( E_\gamma^{lab}, E_\gamma^{nucl} \right) = -1
\end{equation}

\begin{equation}
\label{maxwell_boltzmann_distribution}
w\left(v_\parallel \right) = \sqrt{\frac{M}{2 \pi k_B T}} \cdot \mathrm{exp} \left( -\frac{M v_\parallel^2}{2 k_B T} \right) 
\end{equation}

\begin{equation}
\label{convolution_v}
\sigma^{D}_{0 \to i} (E) = \int_{-c}^{c} \sigma_{0 \to i} (E, E_i^{nucl} \to E_i^{lab}(v_\parallel)) \cdot w(v_\parallel) \mathrm{d} v_\parallel
\end{equation}

\begin{equation}
\label{substitute_v_E}
\mathrm{d} v_\parallel = \left( \frac{\mathrm{d} v_\parallel}{\mathrm{d} E_i^{lab}} \right) \mathrm{d} E_i^{lab} = \frac{- 4 \cdot c \cdot \frac{E_\gamma^{lab}}{\left( E_\gamma^{nucl} \right)^2}}{ \left[ 1 + \left( \frac{E_\gamma^{lab}}{E_\gamma^{nucl}} \right)^2 \right]^2} ~ \mathrm{d} E_i^{lab}
\end{equation}

\begin{equation}
\label{convolution_E_0}
\sigma^{D}_{0 \to i} (E) = \int_{E_\gamma^{lab} \left(-c \right)}^{E_\gamma^{lab} \left(c \right)} \sigma_{0 \to i} (E, E_i^{lab}) \cdot w(E_i^{lab}) \left( \frac{\mathrm{d} v_\parallel}{\mathrm{d} E_i^{lab}} \right) \mathrm{d} E_i^{lab}
\end{equation}

\begin{equation}
\label{convolution_E}
\sigma^{D}_{0 \to i} (E) = - \int_{0}^{\infty} \sigma_{0 \to i} (E, E_i^{lab}) \cdot w(E_i^{lab}) \left( \frac{\mathrm{d} v_\parallel}{\mathrm{d} E_i^{lab}} \right) \mathrm{d} E_i^{lab}
\end{equation}

\newpage

\bibliography{references}{}
\bibliographystyle{plain}
\end{document}
