\documentclass{article}

\usepackage{cite}
\usepackage{hyperref}
\usepackage{amsmath}
\usepackage{mathtools}
\usepackage{bbold}

\begin{document}
\title{SeAN, Self-Absorption Numerical \\ Theoretical background}
\author{Udo Gayer}
\date{}
\maketitle

\newpage

\tableofcontents

\newpage

\section{Introduction}
This document describes the theoretical background of nuclear self-absorption experiments relevant for the usage of SeAN. 
%A comprehensive introduction to the experimental technique can be found in the PhD thesis of C. Romig (Ref. \cite{Rom15}).
%The notation in all experiment - related documents is also based on this reference.

%One of the first introductions into the formalism was given by F.R. Metzger in 1959 \cite{Met59}.

\newpage

\section{Photoabsorption cross section}
Photoabsorption in this context means the absorption of a photon by an atomic nucleus, causing an excitation of the nucleus. A special focus in the derivation of the absorption cross section will be on the inclusion of effects of the presence of the nucleus inside some system of atoms. The energy scales of nuclear excitations and excitations of, for example, a crystal lattice, differ by about a factor of $10^6$. Nevertheless, as will become clear in the following, the binding in a system of atoms can have an impact on a measurement of nuclear observables.
%The separation of scales will be used in several cases where expressions can be simplified.

The discussion is based on the article "Capture of Neutrons by Atoms in a Crystal" by Willis E. Lamb \cite{Lam39}. 
Although Lamb discusses the influence of atomic binding on the absorption of neutrons in particular, the formalism is sufficiently general to allow for the description of photoabsorption as well. In the following text, the notation was sometimes adapted to fit this application or to better illustrate the meaning of variables.

In Ref. \cite{Lam39}, Lamb introduces the probability $W$ for the capture of a projectile particle with the momentum $\mathbf{p}$ by a nucleus $A$ and the subsequent emission of another particle with momentum $\mathbf{k}$ to form a nucleus $B$:

\begin{equation}
\label{absorption_general}
W(A, \alpha_s, B, \beta_s) = \left| \sum_{n_s} \frac{\langle B, \beta_s, \mathbf{k} | H | C, n_s \rangle \langle C, n_s | H | A, \alpha_s, \mathbf{p} \rangle }{E_0 - E + E(n_s) - E(\alpha_s) + i/2 \Gamma(n_s)} \right| ^2
\end{equation}

The transition from nucleus $A$ to $B$ is assumed to proceed via an intermediate nucleus $C$. 
Furthermore, the expression also accounts for the binding of the atoms in a multi-atom system (in the following only called 'the system'). 
The system goes from the initial state $\alpha_s$ to the final state $\beta_s$ via intermediate states $n_s$, where the label $s$ indicates a possible degeneracy of these states. 
The transitions are mediated by an interaction Hamiltonian $H$.
In the denominator, the energy $E$ is the kinetic energy of the projectile, $E_0$ is the resonance energy at rest and neglecting the recoil of the capturing nucleus, and $E(\alpha_s)$ and $E(n_s)$ are the excitation energies of the states of the system by which the reaction proceeds.
Finally, $\Gamma(n_s)$ is the width of the excited state $| C, n_s \rangle$, which is related to its half-life $\tau(n_s)$ like:

\begin{equation}
\label{gamma_tau}
	\Gamma \cdot \tau = \hbar
\end{equation}

A sum is taken over all intermediate states of the system $n_s$ to account for the fact that there is often a quasi-continuum of such states available. On the contrary, no sum is taken over the intermediate nuclear states $C$ since the absorption on an isolated state is to be studied.
The matrix elements

\begin{equation}
\label{me_to_intermediate}
	\langle C, n_s | H | A, \alpha_s, \mathbf{p} \rangle
\end{equation}

and

\begin{equation}
\label{me_to_final}
	\langle B, \beta_s, \mathbf{k} | H | C, n_s \rangle
\end{equation}

describe the excitation of the system and the nucleus from the initial state $| A, \alpha_s \rangle$ to an intermediate state $| C, n_s \rangle$, and the de-excitation to the final state $| B, \beta_s \rangle$.

It is assumed that the interaction consists of an interaction of the nucleus, which is abbreviated as a nuclear matrix element $M^{\mathrm{nucl}}$ and a momentum transfer to the system. 
A momentum transfer is executed by acting with the operator

\begin{equation}
	\hat{O} = \exp{\left(i \mathbf{p} \mathbf{x} / \hbar\right)}
\end{equation}

on the wave function. Here, $\mathbf{p}$ and $\mathbf{x}$ denote the transveral momentum and the position operator.
Using this separation between the interaction with the nucleus and the system, the matrix elements in Eq. \ref{me_to_intermediate} and \ref{me_to_final} can be factorized into:

\begin{equation}
\label{me_to_intermediate_factorized}
\langle n_s | \exp{\left( i \mathbf{p} \cdot \mathbf{x} / \hbar \right)} | \alpha_s \rangle \cdot M^{\mathrm{nucl}}_{A \to C}
\end{equation}

\begin{equation}
\label{me_to_final_factorized}
\langle \beta_s | \exp{\left( - i \mathbf{k} \cdot \mathbf{x} / \hbar \right)} | n_s \rangle \cdot M^{\mathrm{nucl}}_{C \to B}
\end{equation}

Using the factorized matrix elements, Eq. \ref{absorption_general} becomes:

\begin{equation}
\label{absorption_general_factorized}
	W (A, \alpha_s, B, \beta_s)= \left| M^{\mathrm{nucl}}_{C \to B} \right|^2 \cdot \left| M^{\mathrm{nucl}}_{A \to C} \right|^2 \cdot 
\end{equation}

\begin{equation*}
\left| \sum_{n_s} \frac{ \langle \beta_s | \exp{\left( - i \mathbf{k} \cdot \mathbf{x} / \hbar \right)} | n_s \rangle \langle n_s | \exp{\left( i \mathbf{p} \cdot \mathbf{x} / \hbar \right)} | \alpha_s \rangle  }{E_0 - E + E(n_s) - E(\alpha_s) + i/2 \Gamma(n_s)} \right| ^2
\end{equation*}

Here, the nuclear matrix elements have already been factored out of the sum.

In fact, for the discussion of photo\textit{absorption}, the final state of the system and the nucleus are irrelevant. Therefore, consider only the probability for absorption by summing over all final states:

\begin{equation}
\label{absorption_general_no_final}
W(A, \alpha_s) = \sum_{\beta_s} W(A, \alpha_s, B, \beta_s)
\end{equation}


Furthermore, in constrast to the nucleus, the system is often not in a definite state $\alpha_s$ at the time of the capture, but rather distributed among the set $\{ \alpha_s \}$, with the probability $g(\alpha_s)$ of finding the system in the definite state $\alpha_s$:

\begin{equation}
	\label{g_alpha}
	g(\alpha_s) = \langle \alpha_s | \{ \alpha_s \} \rangle
\end{equation}

To account for this, introduce a weighted sum over $\alpha_s$ in Eq. \ref{absorption_general}:

\begin{equation}
\label{absorption_general_alpha_s_distribution}
W = \sum_{\alpha_s} g(\alpha_s) W(A, \alpha_s)
\end{equation}

\begin{equation}
\label{absorption_simplification_1}
	W \propto \sum_{\alpha_s} g(\alpha_s) \sum_{\beta_s} \left| \sum_{n_s} \frac{ \langle \beta_s | \exp{\left( - i \mathbf{k} \cdot \mathbf{x} / \hbar \right)} | n_s \rangle \langle n_s | \exp{\left( i \mathbf{p} \cdot \mathbf{x} / \hbar \right)} | \alpha_s \rangle  }{E_0 - E + E(n_s) - E(\alpha_s) + i/2 \Gamma(n_s)} \right| ^2
\end{equation}

In Eq. \ref{absorption_simplification_1}, the nuclear matrix elements have been left out for sake of readability. 
The following simplifications do not depend on them. 
First, calculate the absolute value explicitly to be able to exchange the order of the summations:

\begin{equation}
\label{absorption_simplification_2}
	W \propto \sum_{\alpha_s} g(\alpha_s) \sum_{\beta_s} \sum_{n_s} \frac{ \langle \beta_s | \exp{\left( - i \mathbf{k} \cdot \mathbf{x} / \hbar \right)} | n_s \rangle \langle n_s | \exp{\left( i \mathbf{p} \cdot \mathbf{x} / \hbar \right)} | \alpha_s \rangle  }{E_0 - E + E(n_s) - E(\alpha_s) + i/2 \Gamma(n_s)} 
\end{equation}	
\begin{equation*}
\sum_{m_s} \frac{ \langle m_s | \exp{\left( i \mathbf{k} \cdot \mathbf{x} / \hbar \right)} | \beta_s \rangle \langle \alpha_s | \exp{\left(  - i \mathbf{p} \cdot \mathbf{x} / \hbar \right)} | n_s \rangle  }{E_0 - E + E(n_s) - E(\alpha_s) - i/2 \Gamma(n_s)}
\end{equation*}

Since all sums run over the whole space, they can be interchanged. 
The same holds for the matrix elements which are simply numbers:

\begin{equation}
\label{absorption_simplification_3}
	W \propto \sum_{\alpha_s} g(\alpha_s) \sum_{n_s} \sum_{m_s} \frac{  \langle n_s | \exp{\left( i \mathbf{p} \cdot \mathbf{x} / \hbar \right)} | \alpha_s \rangle \langle \alpha_s | \exp{\left(  - i \mathbf{p} \cdot \mathbf{x} / \hbar \right)} | n_s \rangle  }{E_0 - E + E(n_s) - E(\alpha_s) + i/2 \Gamma(n_s)} 
\end{equation}	
\begin{equation*}
	\frac{ \langle m_s | \exp{\left( i \mathbf{k} \cdot \mathbf{x} / \hbar \right)} \sum_{\beta_s} \left( | \beta_s \rangle \langle \beta_s | \right) \exp{\left( - i \mathbf{k} \cdot \mathbf{x} / \hbar \right)} | n_s \rangle }{E_0 - E + E(n_s) - E(\alpha_s) - i/2 \Gamma(n_s)}
\end{equation*}

Using the completeness relation

\begin{equation}
	\label{completeness_relation_system}
	\sum_{\beta_s} | \beta_s \rangle \langle \beta_s | = \mathbb{1}
\end{equation}

for the states $\beta_s$ the numerator in the second line of Eq. \ref{absorption_simplification_3} can be turned into a delta function in $m_s$ and $n_s$:

\begin{equation}
	\label{absorption_simplification_4}
	\underbrace{\langle m_s | \underbrace{\exp{\left( i \mathbf{k} \cdot \mathbf{x} / \hbar \right)} \underbrace{\sum_{\beta_s} \left( | \beta_s \rangle \langle \beta_s | \right)}_{ = \mathbb{1}} \exp{\left( - i \mathbf{k} \cdot \mathbf{x} / \hbar \right)} }_{ = \mathbb{1}} | n_s \rangle }_{ = \delta_{m_s, n_s}}
\end{equation}

As expected when the final states are of no interest, the momentum $\mathbf{k}$ of the emitted particle drops out as well. 
Now, the absolute values in the numerator and the denominator can be calculated separately and the final probability for absorption is:

\begin{equation}
\label{absorption_general_summed}
	W = \left| M^{\mathrm{nucl}}_{C \to B} \right|^2 \cdot \left| M^{\mathrm{nucl}}_{A \to C} \right|^2 \sum_{\alpha_s} g(\alpha_s) \sum_{n_s} \frac{ \left| \langle n_s | \exp{\left( i \mathbf{p} \cdot \mathbf{x} / \hbar \right)} | \alpha_s \rangle \right|^2  }{\left[ E - E_0 - \left( E(n_s) - E(\alpha_s) \right) \right]^2 + 1/4 \left( \Gamma(n_s) \right)^2 }
\end{equation}

Coming from the more general Eq. \ref{absorption_general}, Eq. \ref{absorption_general_summed} has now been adapted for the photoabsorption experiments which are considered here.
Note that $W$ is a \textit{probability} for the absorption and emission process. To adapt to the case of photoabsorption and emission, the matrix elements have to be replaced by \cite{Bet37}

\begin{equation}
	\label{probability_to_cross_section}
	\left| M^{\mathrm{nucl}}_{C \to B} \right|^2 \cdot \left| M^{\mathrm{nucl}}_{A \to C} \right|^2 = \pi \left( \frac{\hbar c}{E_j - E_i} \right)^2 \frac{2 J_j + 1}{2 J_i + 1} \Gamma_i \Gamma_j
\end{equation}

In Eq. \ref{probability_to_cross_section}, $i$ denotes the initial state of the nucleus (mostly the ground state) and j the excited state at an energy $E_j$. The nuclear states have a corresponding angular momentum quantum number $J$ and a partial transition width $\Gamma$ which is a function of the transition matrix element.

Usually, the nuclear matrix elements $M^{\mathrm{nucl}}$ are the quantities to be measured, so they will be left as "black boxes" in the following. 
Of course, a good estimate of the magnitude of the nuclear matrix element is crucial for the planning of an experiment.

In the following, different methods of including the influence of the system on the resonant absorption will be described. First, however, it is instructive to discuss the case of capture on a free atom.

\section{Free atom}
The wave function of a free atom can be modeled as a plane wave:

\begin{equation}
\label{plane_wave_unnormalized}
	\langle x | \alpha_s \rangle \propto \exp{ \left( i \mathbf{k}_{\alpha_s} \cdot \mathbf{x} / \hbar \right) }
\end{equation}

The symbol for proportionality was used in Eq. \ref{plane_wave_unnormalized}, because in the complete $\rm I\!R^3$ space, a plane wave can not be normalized explicitly. 
This problem is usually overcome by treating the problem in a finite box with side lengths $L$ and volume $L^3$, where the wave functions can be normalized:

\begin{equation}
\label{plane_wave_normalized}
	\langle x | \alpha_s \rangle = \frac{1}{L^{ 3/2}} \exp{ \left( i \mathbf{k}_{\alpha_s} \cdot \mathbf{x} / \hbar \right) }
\end{equation}

This imposes a restriction on the possible values for the momentum $\mathbf{k}_{\alpha_s}$:

\begin{equation}
	\label{k_quantization}
	k_{\alpha_s} = \frac{\hbar \pi}{L} \left( n_x \mathbf{e}_x + n_y \mathbf{e}_y + n_z \mathbf{e}_z \right)
\end{equation}

\begin{equation*}
	n_i \in \mathbb{Z}
\end{equation*}

%At the end of a calculation in the finite space, taking the limit $L \to \infty$ should be possible if the calculated observable is independent of the size of the space. 
Using plane waves, considering that $\mathbf{p}$ is not an operator but a constant vector, and inserting unity operators in real space

\begin{equation}
	\int_{-L/2}^{L/2} \mathrm{d}^3 x' \left| x' \rangle \langle x' \right| = \mathbb{1}
\end{equation}

the matrix element in the numerator in Eq. \ref{absorption_general_summed} becomes:

\begin{equation}
	\label{plane_wave_delta}
	\left| \langle n_s | \exp{\left( i \mathbf{p} \cdot \mathbf{x} / \hbar \right)} | \alpha_s \rangle \right|^2 
\end{equation}
\begin{equation*}
	= \langle n_s | \exp{\left( i \mathbf{p} \cdot \mathbf{x} / \hbar \right)} | \alpha_s \rangle \langle \alpha_s | \exp{\left( -i \mathbf{p} \cdot \mathbf{x} / \hbar \right)} | n_s \rangle
\end{equation*}
\begin{equation*}
	= \int_{-L/2}^{L/2} \frac{\mathrm{d}^3 x}{L^3} \exp{ \left( i \left[ \left( \mathbf{k}_{\alpha_s} + \mathbf{p} \right) - \mathbf{k}_{n_s} \right] \mathbf{x} \right) } \int_{-L/2}^{L/2}\frac{\mathrm{d}^3 x'}{L^3} \exp{ \left( - i \left[ \mathbf{k}_{n_s} - \left( \mathbf{k}_{\alpha_s} + \mathbf{p} \right) \right] \mathbf{x}' \right) }
\end{equation*}
\begin{equation*}
	= \delta \left( \left[ \mathbf{k}_{\alpha_s} + \mathbf{p} \right] - \mathbf{k}_{n_s} \right)^2
\end{equation*}

In the last step, a definition of the Dirac $\delta$-function has been used:

\begin{equation}
	\int_{-L/2}^{L_2} \exp{ \left( i a x \right) } = 
	\begin{cases}
	L^3, a = 0,\\
	0 \text{, else (residual theorem)}
	\end{cases}
\end{equation}

Since all the dimension $L$ of the space vanishes in the final result, it is justified to take the limit to infinity.

What the $\delta$-function does is obvious: it ensures the conservation of momentum when a photon with a momentum $\mathbf{p}$ is absorbed. 
Therefore, the capture on a free atom can be treated as an inelastic process between two particles. 

For the photon and the nucleus $^A_ZX$, which has an excited state $^A_ZX^*$ at an energy $E_0$, the 4-momenta before the capture are:

\begin{equation}
\label{four_momentum_photon}
	p^\mu_{\mathrm{ph}} = \left( p c, \mathbf{p} \right)
\end{equation}
\begin{equation}
\label{four_momentum_nucleus}
	p^\mu_{\mathrm{X}} = \left( E_{\alpha_s} = \sqrt{ \left(Au c^2\right)^2 + k_{\alpha_s}^2 c^2}, \mathbf{k}_{\alpha_s} \right)
\end{equation}

In Eq. \ref{four_momentum_nucleus}, the atomic mass unit $u$ has been used.
After the capture, the 4-momentum of the excited nucleus is:

\begin{equation}
	\label{four_momentum_excited_nucleus}
	p^\mu_{\mathrm{X^*}} = \left( E_{n_s} = \sqrt{ \left( Au c^2 + E_0 \right)^2 + k_{n_s}^2 c^2}, \mathbf{k}_{n_s} \right)
\end{equation}

Using the conservation of momentum and solving the conservation of energy

\begin{equation}
	\label{conservation_of_energy}
	p^0_\mathrm{ph} + p^0_\mathrm{X} = p c + \sqrt{ Au c^2 + k_{\alpha_s}^2 c^2} = \sqrt{ \left( Au c^2 + E_0 \right)^2 + (\mathbf{k}_{\alpha_s} + \mathbf{p})^2 c^2} = p^0_\mathrm{X^*}
\end{equation}

for the energy of the photon $E_\mathrm{ph} = pc$, one gets:

\begin{equation}
	\label{recoil_energy}
	E_\mathrm{ph} = \frac{E_0 \cdot Auc^2 + E_0^2/2}{\sqrt{\left( Auc^2\right)^2 + k_{\alpha_s}^2 c^2} - k_{\alpha_s}c}
\end{equation}

The equation is easier to interpret when the nucleus can be considered to be at rest ($k_{\alpha_s} = 0$) before the capture. 
Then, Eq. \ref{recoil_energy} becomes:

\begin{equation}
	\label{recoil_energy_at_rest}
	E_\mathrm{ph} = E_0 \cdot \left( 1 + \frac{E_0}{2 Auc^2} \right)
\end{equation}

Eq. \ref{recoil_energy_at_rest} states that the photon needs to have not only enough energy to excite the state $j$, but the recoil energy that is transferred to the nucleus also needs to be considered.

Putting the considerations that were made up to now together, Eq. \ref{absorption_general_summed} becomes

\begin{equation}
\label{absorption_general_plane_wave}
	W = \left| M^{\mathrm{nucl}}_{C \to B} \right|^2 \cdot \left| M^{\mathrm{nucl}}_{A \to C} \right|^2 \int \mathrm{d}^3 k_{\alpha_s}  \frac{g(\alpha_s)}{ \left[ E' (E, \mathbf{k}_{\alpha_s})- E_0 - \frac{E_0^2}{2 Auc^2} \right]^2 + \Gamma^2 / 4}
\end{equation}

in the reference frame where the nucleus is at rest. 
The transition to the rest frame requires that the energy of the photon $E$ is Doppler-shifted.
This is why in Eq. \ref{absorption_general_plane_wave}, $E'$ depends on $E$ and the initial momentum of the nucleus in the laboratory frame.

%\begin{equation}
%\label{absorption_general_plane_wave}
%	W = \left| M^{\mathrm{nucl}}_{C \to B} \right|^2 \cdot \left| M^{\mathrm{nucl}}_{A \to C} \right|^2 \int \mathrm{d}^3 k_{\alpha_s}  \frac{g(\alpha_s)}{ \left[ E - E_0 (\mathbf{k}_{\alpha_s}) - \frac{E_0^2}{2 Auc^2} \right]^2 + \Gamma^2 / 4}
%\end{equation}
%
%in the reference frame where the nucleus is at rest. 
%The transition to the rest frame requires that the transition energy $E_0$ is Doppler-shifted.
%This is why in Eq. \ref{absorption_general_plane_wave}, $E_0$ depends on $\mathbf{k}_{\alpha_s}$, the momentum of the nucleus in the laboratory frame.

Instead of the momentum $\mathbf{k}$, we will now use the velocity $v$, and get a more common expression for the relativistic Doppler shift:

%\begin{equation}
%	\label{doppler_shifted_transition_energy}
%	E_0 (\mathbf{v}_{\alpha_s}) = \frac{\sqrt{1 - \left( \frac{v_{\alpha_s, \parallel}}{c} \right)^2}}{1 + \frac{v_{\alpha_s, \parallel}}{c}} \cdot E_0
%\end{equation}

\begin{equation}
	\label{doppler_shifted_transition_energy}
	E' (E, \mathbf{v}_{\alpha_s}) = \frac{1 + \frac{v_{\alpha_s, \parallel}}{c}}{\sqrt{1 - \left( \frac{v_{\alpha_s, \parallel}}{c} \right)^2}} \cdot E \approx \left( 1 + \frac{v_{\alpha_s, \parallel}}{c} \right) \cdot E
\end{equation}

The index $\parallel$ of the velocity means that only the projection of the velocity of the nucleus on the momentum axis of the photon is responsible for the Doppler shift:

\begin{equation}
	\label{v_parallel}
	v_{\alpha_s, \parallel} = \mathbf{v}_{\alpha_s} \cdot \frac{\mathbf{p}}{\left| \mathbf{p} \right|}
\end{equation}

The approximation in Eq. \ref{doppler_shifted_transition_energy} holds in the non-relativistc limit of low velocities, which is well justified for free atoms at "normal" temperatures.

In Eq. \ref{absorption_general_plane_wave}, the average over a distribution of momentum states can be replaced by an integration over all possible velocities:

\begin{equation}
\label{absorption_plane_wave_v}
	W = \left| M^{\mathrm{nucl}}_{C \to B} \right|^2 \cdot \left| M^{\mathrm{nucl}}_{A \to C} \right|^2 \int \mathrm{d} v_{\alpha_s, \parallel}  \frac{g(v_{\alpha_s, \parallel})}{ \left[ E' (E, v_{\alpha_s, \parallel})- E_0 - \frac{E_0^2}{2 Auc^2} \right]^2 + \Gamma^2 / 4}
\end{equation}

The distribution of velocities of free atoms in thermal equilibrium at a temperature T is a Maxwell-Boltzmann distribution:

\begin{equation}
	\label{maxwell_boltzmann_distribution}
	g(v_\parallel) \mathrm{d} v_\parallel = \sqrt{\frac{A u}{2 \pi k_B T}} \exp{\left( - \frac{A u v_\parallel^2}{2 k_B T} \right)} \mathrm{d} v_\parallel
\end{equation}

\section{Approximations for atoms in solids}

\subsection{Effective temperature}

%\subsubsection{Debye approximation}

%\subsubsection{Effective temperature from phonon densities}

%\subsection{Harmonic crystal}

%\subsection{Arbitrary system of atoms}

\section{Computational considerations}

\begin{equation}
\label{breit_wigner}
\sigma_{0 \to i} (E, E_i^{nucl}) = \frac{\pi}{2} \cdot \left( \frac{\hbar c}{E_i^{nucl}} \right)^2 \cdot \frac{2 J_i + 1}{2 J_0 + 1} \cdot \frac{\Gamma_0 \Gamma}{\left( E - E_i^{nucl} \right)^2 + \frac{\Gamma^2}{4}}
\end{equation}

\begin{equation}
\label{doppler_shift}
E_\gamma^{lab}(v_\parallel) = \frac{\sqrt{1 - \left( \frac{v_\parallel}{c} \right)^2}}{1 + \frac{v_\parallel}{c}} \cdot E_\gamma^{nucl}
\end{equation}

\begin{equation}
\label{doppler_shift_inverse}
\frac{v_\parallel}{c} \left( E_\gamma^{lab}, E_\gamma^{nucl} \right) = \frac{1 - \left( \frac{E_\gamma^{lab}}{E_\gamma^{nucl}} \right)^2}{1 + \left( \frac{E_\gamma^{lab}}{E_\gamma^{nucl}} \right)^2}
\end{equation}

\begin{equation}
\frac{v_\parallel}{c} \left( 0, E_\gamma^{nucl} \right) = 1
\end{equation}

\begin{equation}
	\frac{v_\parallel}{c} \left( E_\gamma^{nucl}, E_\gamma^{nucl} \right) = 0
\end{equation}

\begin{equation}
\mathrm{lim}_{E_\gamma^{lab} \to \infty} \frac{v_\parallel}{c} \left( E_\gamma^{lab}, E_\gamma^{nucl} \right) = -1
\end{equation}

\begin{equation}
\label{maxwell_boltzmann_distribution}
w\left(v_\parallel \right) = \sqrt{\frac{M}{2 \pi k_B T}} \cdot \mathrm{exp} \left( -\frac{M v_\parallel^2}{2 k_B T} \right) 
\end{equation}

\begin{equation}
\label{pseudo_convolution_v}
\sigma^{D}_{0 \to i} (E) = \int_{-c}^{c} \sigma_{0 \to i} (E, E_i^{nucl} \to E_i^{lab}(v_\parallel)) \cdot w(v_\parallel) \mathrm{d} v_\parallel
\end{equation}

\begin{equation}
\label{substitute_v_E}
\mathrm{d} v_\parallel = \left( \frac{\mathrm{d} v_\parallel}{\mathrm{d} E_i^{lab}} \right) \mathrm{d} E_i^{lab} = \frac{- 4 \cdot c \cdot \frac{E_\gamma^{lab}}{\left( E_\gamma^{nucl} \right)^2}}{ \left[ 1 + \left( \frac{E_\gamma^{lab}}{E_\gamma^{nucl}} \right)^2 \right]^2} ~ \mathrm{d} E_i^{lab}
\end{equation}

\begin{equation}
\label{pseudo_convolution_E_0}
\sigma^{D}_{0 \to i} (E) = \int_{E_\gamma^{lab} \left(-c \right)}^{E_\gamma^{lab} \left(c \right)} \sigma_{0 \to i} (E, E_i^{lab}) \cdot w(E_i^{lab}) \left( \frac{\mathrm{d} v_\parallel}{\mathrm{d} E_i^{lab}} \right) \mathrm{d} E_i^{lab}
\end{equation}

\begin{equation}
\label{pseudo_convolution_E}
\sigma^{D}_{0 \to i} (E) = - \int_{0}^{\infty} \sigma_{0 \to i} (E, E_i^{lab}) \cdot w(E_i^{lab}) \left( \frac{\mathrm{d} v_\parallel}{\mathrm{d} E_i^{lab}} \right) \mathrm{d} E_i^{lab}
\end{equation}

\begin{equation}
\label{convolution_f_1}
\sigma_{0 \to i} (E, E_i^{lab}) = \frac{\pi}{2} \cdot \left( \frac{\hbar c}{E_i^{nucl}} \right)^2 \cdot \frac{2 J_i + 1}{2 J_0 + 1} \cdot \frac{\Gamma_0 \Gamma}{\left( E - E_i^{nucl} \right)^2 + \frac{\Gamma^2}{4}} 
\end{equation}
\begin{equation}
= \frac{1}{\left( E_i^{lab} \right)^2} \cdot f\left( E - E^{lab}_i \right)
\end{equation}

\begin{equation}
\label{convolution_g}
	w \left( E^{lab}_i \right)\left( \frac{\mathrm{d} v_\parallel}{\mathrm{d} E_i^{lab}} \right) \coloneqq g \left( E^{lab}_i \right)
\end{equation}

\begin{equation}
\label{convolution_approximation_1}
\sigma^{D}_{0 \to i} (E) = - \int_{0}^{\infty} \frac{1}{\left( E_i^{lab} \right)^2} \cdot f\left( E - E^{lab}_i \right) \cdot g \left( E^{lab}_i \right) \mathrm{d} E_i^{lab}
\end{equation}

\begin{equation}
\label{convolution_approximation_2}
\approx - \int_{0}^{\infty} \frac{1}{\left( E_i^{nucl} \right)^2} \cdot f\left( E - E^{lab}_i \right) \cdot g \left( E^{lab}_i \right) \mathrm{d} E_i^{lab}
\end{equation}

\begin{equation}
\label{convolution_approximation_3}
	\approx - \int_{0}^{\infty} \tilde{f}\left( E - E^{lab}_i \right) \cdot g \left( E^{lab}_i \right) \mathrm{d} E_i^{lab} = -\left( f * g \right) \left( E \right)
\end{equation}

\begin{equation}
\label{discrete_convolution}
	\sigma^{D}_{0 \to i} (E) = \sum_n \tilde{f} \left( E -  \left(E^{lab}_i\right)_n \right) \cdot g \left( \left( (E^{lab}_i\right)_n \right)
\end{equation}
\newpage

\bibliography{references}{}
\bibliographystyle{unsrt}
\end{document}
